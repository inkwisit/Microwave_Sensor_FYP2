\chapter{INTRODUCTION}
Short Range Microwave radar \cite{strohm2005development} is a continuously growing field of research in field of robotics and automotive. These microwave radar are being actively used to develop highly reliable sensors for Advanced Driver Assistance System (ADAS) in Automotive industry. The signal used for finding ranges in radar systems is an electromagnetic (EM) wave at microwave frequencies. The main advantage of radar systems compared to other altenatives such as sonar and lidar is the immunity to weather conditions and potential for lower cost realization. It requires a directional radar, RF front-end, compact \& low-power DSP computer to implement the sensor \cite{charvatres}. 
\section{Motivation}
Adaptive cruise control, navigation, search and surveillance, high resolution imaging and mapping, space flight, and sounding are the current applications worldwide. There are other many sensors available like stereo-camera which can provide visual data. However for the application of aforesaid navigation and cruise control it would require a lot computation effort to determine the distance of objects in responsive amount of time. With visual data, some non-reliable algorithmic process has to be followed which gives visual depth with highest "probability" of accuracy. When deployed in practical environments, the visual data in highly unreliable. Alternatively a LiDAR sensor can be used but it may not work during fog, mist etc. Another safety concern is the car’s ability to detect lightly-colored objects in real-time, such as nearby white vehicles.\\	
	A directional electromagnetic sensor is a better option which can not only provide distance of objects in front of it but also the angular position of the objects. In LiDAR, to detect angular position a single sensor has to be mechanically rotated. The mechanical rotation limits the speed at which it can sweep the area to detect targets, hence degrade agility of the vehicle. If a array of LiDAR sensor is used, it would take a lot of space and power. However the LiDAR sensor can be used to provide an additional safety layer. \\
	We want to develop our own design of such compact radar sensors with anticipated improvement in its angular resolution. In addition, it would allow to get  knowledge of Signal Analysers, Vector Network Analysers, FPGAs (for digital signal processing), High-frequency PCB designs etc.
\section{Present State of the Art}
Recently in 2017, industry leaders demonstrated recently developed small integrated sensors (AWR/IWR series from Texas Instruments Inc., Drive360 from Analog Devices Inc., MR2001 from NXP Semiconductors and more). All the principle components are integrated in a single silicon chip (in case of AWR) or in a single PCB presenting a very comfortable way for automotive engineers to test and deploy these sensors. However for more custom applications where modifications are required for better functionality it is required to analyse the present state of the art and modify/re-develop if required.
\section{Problem Statement}
Our project goal is development of compact frequency modulated continous wave radar for navigation and cruise control of unmanned ground vehicles. The design is done to improve the angular resolution of the system. The objectives are listed as following :
\begin{enumerate}
\item Simulating all possible modules in simulation software (CST or HFSS).
\item Development of individual modules in Print Circuit Board and thorough testing.
\item Integrating all the modules to form a complete PCB radar.
\item Demonstration.
\end{enumerate}
\section{Proposed Solution and Design Effort}
	The configuration of Radar would be Bistatic and it would use continous wave. As opposed to a continous wave radar, a pulse radar uses a complicated matched filter which requires not only complex electronic systems but also capital. We are making a bistatic radar to avoid duplexer. A duplexer is also a complicated device requiring great amount of isolation between ports. Any appreciable mismatch can cause damage to the sensor. Taking into account of modern microstrip antenna size, it is more safe and easy to make a bistatic radar than a monostatic radar. \\
	The frequency range of operation we have selected is S-band. At this range all the Integrated-Circuits are available at low cost. This is due to existance of ISM band (2.4 GHz) in S-band which covers wide variety products available in market. For compactness we would use microstrip antennas. It would be simulated using HFSS then fabricated in a PCB.
\section{Report Organisation}